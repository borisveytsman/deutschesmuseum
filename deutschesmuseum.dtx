% \iffalse
%
% Copyright 2025, Boris Veytsman
% This work may be distributed and/or modified under the
% conditions of the LaTeX Project Public License, either
% version 1.3 of this license or (at your option) any 
% later version.
% The latest version of the license is in
%    http://www.latex-project.org/lppl.txt
% and version 1.3 or later is part of all distributions of
% LaTeX version 2005/12/01 or later.
%
% This work has the LPPL maintenance status `maintained'.
%
% The Current Maintainer of this work is Boris Veytsman,
% <borisv@lk.net> 
%
% This work consists of the file deutschesmuseum.dtx and the
% derived file deutschesmuseum.cls
%
% \fi 
% \CheckSum{0}
%
%
%% \CharacterTable
%%  {Upper-case    \A\B\C\D\E\F\G\H\I\J\K\L\M\N\O\P\Q\R\S\T\U\V\W\X\Y\Z
%%   Lower-case    \a\b\c\d\e\f\g\h\i\j\k\l\m\n\o\p\q\r\s\t\u\v\w\x\y\z
%%   Digits        \0\1\2\3\4\5\6\7\8\9
%%   Exclamation   \!     Double quote  \"     Hash (number) \#
%%   Dollar        \$     Percent       \%     Ampersand     \&
%%   Acute accent  \'     Left paren    \(     Right paren   \)
%%   Asterisk      \*     Plus          \+     Comma         \,
%%   Minus         \-     Point         \.     Solidus       \/
%%   Colon         \:     Semicolon     \;     Less than     \<
%%   Equals        \=     Greater than  \>     Question mark \?
%%   Commercial at \@     Left bracket  \[     Backslash     \\
%%   Right bracket \]     Circumflex    \^     Underscore    \_
%%   Grave accent  \`     Left brace    \{     Vertical bar  \|
%%   Right brace   \}     Tilde         \~} 
%
%\iffalse
% Taken from xkeyval.dtx
%\fi
%\makeatletter
%\def\DescribeOption#1{\leavevmode\@bsphack
%              \marginpar{\raggedleft\PrintDescribeOption{#1}}%
%              \SpecialOptionIndex{#1}\@esphack\ignorespaces}
%\def\PrintDescribeOption#1{\strut\emph{option}\\\MacroFont #1\ }
%\def\SpecialOptionIndex#1{\@bsphack
%    \index{#1\actualchar{\protect\ttfamily#1}
%           (option)\encapchar usage}%
%    \index{options:\levelchar#1\actualchar{\protect\ttfamily#1}\encapchar
%           usage}\@esphack}
%\def\DescribeOptions#1{\leavevmode\@bsphack
%  \marginpar{\raggedleft\strut\emph{options}%
%  \@for\@tempa:=#1\do{%
%    \\\strut\MacroFont\@tempa\SpecialOptionIndex\@tempa
%  }}\@esphack\ignorespaces}
%\makeatother
%
% \MakeShortVerb{|}
% \GetFileInfo{deutschesmuseum.dtx}
% \newcommand{\progname}[1]{\textsf{#1}}
% \title{\LaTeX{} Style For \emph{Deutsches Museum}
%   \thanks{\copyright 2025, Boris Veytsman.  This work has been
%   commissioned by Wayne Shahnke}}
% \author{Boris Veytsman\thanks{%
% \href{mailto:borisv@lk.net}{\texttt{borisv@lk.net}},
% \href{mailto:boris@varphi.com}{\texttt{boris@varphi.com}}}} 
% \date{\filedate, \fileversion}
% \maketitle
% \begin{abstract}
%   This package provides class for typesetting books for Deutsches
%   Museum, \url{https://www.deutsches-museum.de}
% \end{abstract}
% \tableofcontents
%
% \clearpage
%
%
%\section{Introduction}
%\label{sec:intro}
%
% Deutsches museum regularly publishes studies.  This package provides
% the way to the authors to format the contributions in \LaTeX.  
%
%
%\section{Options}
%\label{sec:ug_options}
%
% \DescribeOptions{cfonts,nocfonts}%
% The contributions to Deutsches Museum should use their fonts:
% Garamond Berthold BQ and Futura PT.  You probably want to buy them
% and install.  If you do not have them, use the options |nocfonts| to
% substitute them for free fonts available in standard \TeX\
% distriutions.  The option |cfonts| (the default) uses commercial
% fonts.
%
% In both cases the package uses fonts in OTF formats, so you need to
% use Lua\TeX\ or Xe\TeX\ for the compilation.
%
% If you use commercial fonts, the package expects the file names
% listed in Table~\ref{tab:fonts}.
%
% \begin{table}
%   \centering
%   \begin{tabularx}{\textwidth}{>{\raggedright}Xl}
%       \toprule
%       Font & File\\
%       \midrule
%       Garamond BQ Regular & |GaramondBQ-Regular.otf|\\
%       Garamond BQ Italic & |GaramondBQ-Italic.otf| \\
%       Garamond BQ Bold & |GaramondBQ-Bold.otf|\\
%       Garamond BQ Medium Italic & |GaramondBQ-MediumItalic.otf|\\
%       Futura PT Light & |futura-pt_light.otf|\\
%       Futura PT Light Oblique & |futura-pt_light-oblique.otf|\\
%       Futura PT Medium & |futura-pt_medium.otf|\\
%       Futura PT Medium Oblique & |futura-pt_medium-oblique.otf|\\
%       Futura PT Condensed Medium & |futura-pt_cond-medium.otf|\\
%       Futura PT Condensed Medium Oblique & |futura-pt_cond-medium-oblique.otf|\\
%       \bottomrule
%     \end{tabularx}
%   \caption{Commerical fonts (if the option \texttt{cfonts} is used)}
%   \label{tab:fonts}
% \end{table}
%
% Internally the class uses \textsl{amsart}, so all its options are
% supported.  
%
%\section{Front matter}
%\label{sec:ug_frontmatter}
%
% \DescribeMacro{\title}%
% As in \textsl{amsart} package, the \cs{title} command has an
% optional argument, that defines \cs{shortitle} macro, used for
% running heads.
%
% \DescribeMacro{\subtitle}%
% Deutsches Museum contributions, along with the title, have
% |\subtitle|.  The running heads pick either short title, or
% subtitle, whichever is defined last.  You can override this choice
% by redefining |\shortitle|.  Examples
%
% \DescribeMacro{abstract}%
% Abstract, if present, should follow \cs{maketitle}
%
%\section{Bibliography}
%\label{sec:ug_biblio}
%
% The package uses bibliography in footnotes and |natbib| for
% formatting the bibliography list.  The command \cs{cite} produces
% footnote citations. 
%
%
% \StopEventually{%
% \clearpage}
% 
% \clearpage
%
%
%\section{Implementation}
%\label{sec:impl}
%
%\subsection{Identification}
%\label{sec:ident}
%
% We start with the declaration who we are.  Most |.dtx| files put
% driver code in a separate driver file |.drv|.  We roll this code into the
% main file, and use the pseudo-guard |<gobble>| for it.
%    \begin{macrocode}
%<class>\NeedsTeXFormat{LaTeX2e}
%<*gobble>
\ProvidesFile{deutschesmuseum.dtx}
%</gobble>
%<class>\ProvidesClass{deutschesmuseum}
[2025/09/04 v1.1 Typesetting books for Deutsches Museum]
%    \end{macrocode}
%
% And the driver code:
%    \begin{macrocode}
%<*gobble>
\documentclass{ltxdoc}
\usepackage{booktabs,tabularx}
\usepackage[breaklinks,colorlinks,linkcolor=black,citecolor=black,
pagecolor=black,urlcolor=black,hyperindex=false]{hyperref}
\PageIndex
\CodelineIndex
\RecordChanges
\EnableCrossrefs
\begin{document}
  \DocInput{deutschesmuseum.dtx}
\end{document}
%</gobble> 
%<*class>
%    \end{macrocode}
%   
%
%\subsection{Options}
%\label{sec:options}
%
%\begin{macro}{\ifdeutschesmuseum@cfonts}
% First, let us decide whether we have non-free fonts:
%    \begin{macrocode}
\newif\ifdeutschesmuseum@cfonts
\deutschesmuseum@cfontstrue
\DeclareOption{cfonts}{\deutschesmuseum@cfontstrue}
\DeclareOption{nocfonts}{\deutschesmuseum@cfontsfalse}
%    \end{macrocode} 
% \end{macro}
%
% The size-changing options produce a warning:
%    \begin{macrocode}
\long\def\deutschesmuseum@size@warning#1{%
  \ClassWarning{deutschesmuseum}{Size-changing option #1 will not be
    honored}}%
\DeclareOption{8pt}{\deutschesmuseum@size@warning{\CurrentOption}}%
\DeclareOption{9pt}{\deutschesmuseum@size@warning{\CurrentOption}}%
\DeclareOption{10pt}{\deutschesmuseum@size@warning{\CurrentOption}}%
\DeclareOption{11pt}{\deutschesmuseum@size@warning{\CurrentOption}}%
\DeclareOption{12pt}{\deutschesmuseum@size@warning{\CurrentOption}}%
%    \end{macrocode}
% 
%
% All other options are passed to \progname{amsart}:
%    \begin{macrocode}
\DeclareOption*{\PassOptionsToClass{\CurrentOption}{amsart}}
\ProcessOptions\relax
%    \end{macrocode}
%
%\subsection{Loading Class and Packages}
%\label{sec:loading}
%
% We start with the base class
%    \begin{macrocode}
\LoadClass{amsart}
\RequirePackage[breaklinks,colorlinks,linkcolor=black,citecolor=black,
pagecolor=black,urlcolor=black,hyperindex=false]{hyperref}
\RequirePackage{multicol}
%    \end{macrocode}
%
%
%\subsection{Fonts}
%\label{sec:fonts}
%
%    \begin{macrocode}
\usepackage{fontspec}
\ifdeutschesmuseum@cfonts
  \setmainfont{GaramondBQ}[
  Extension = .otf,
  UprightFont = *-Regular,
  BoldFont    = *-Bold,
  ItalicFont  = *-Italic,
  BoldItalicFont = *-MediumItalic]
  \newfontfamily\FuturaPTLight{futura-pt_light}[
  Extension = .otf,
  UprightFont = *,
  ItalicFont = *-oblique,
  BoldFont = futura-pt_medium,
  BoldItalicFont = futura-pt_medium-oblique]
  \newfontfamily\FuturaPTMedium{futura-pt_medium}[
  Extension = .otf,
  UprightFont = *,
  ItalicFont = *-oblique]
  \newfontfamily\FuturaPTCondMedium{futura-pt_cond-medium}[
  Extension = .otf,
  UprightFont = *,
  ItalicFont = *-oblique]
  \newcommand{\foliofont}{\FuturaPTLight\fontsize{9pt}{11pt}\selectfont}
  \newcommand{\hIfont}{\FuturaPTCondMedium\fontsize{14pt}{16pt}\selectfont}
  \newcommand{\hIIfont}{\FuturaPTCondMedium\fontsize{12.5pt}{12pt}\selectfont}
  \newcommand{\hIIIfont}{\FuturaPTCondMedium\fontsize{11pt}{12pt}\selectfont}
  \newcommand{\hIVfont}{\FuturaPTCondMedium\fontsize{10pt}{12pt}\selectfont}
  \newcommand{\fnmarkfont}{\FuturaPTMedium\fontsize{10pt}{11pt}\selectfont}
  \newcommand{\fnmarktextfont}{\FuturaPTMedium\fontsize{8pt}{10pt}\selectfont}
  \newcommand{\fnfont}{\FuturaPTLight\fontsize{8pt}{10pt}\selectfont}
\else
  \setmainfont{EBGaramond}[
  Extension = .otf,
  UprightFont = *-Regular,
  BoldFont    = *-Bold,
  ItalicFont  = *-Italic,
  BoldItalicFont = *-BoldItalic]
  \newfontfamily\LatoLight{Lato-Light}[
  Extension = .ttf,
  UprightFont = *,
  ItalicFont = *Italic,
  BoldFont = Lato-Medium,
  BoldItalicFont = Lato-MediumItalic]
  \newfontfamily\LatoMedium{Lato-Medium}[
  Extension = .ttf,
  UprightFont = *,
  ItalicFont = *Italic]
  \newcommand{\foliofont}{\LatoLight\fontsize{9pt}{11pt}\selectfont}
  \newcommand{\hIfont}{\LatoMedium\fontsize{14pt}{16pt}\selectfont}
  \newcommand{\hIIfont}{\LatoMedium\fontsize{12.5pt}{12pt}\selectfont}
  \newcommand{\hIIIfont}{\LatoMedium\fontsize{11pt}{12pt}\selectfont}
  \newcommand{\hIVfont}{\LatoMedium\fontsize{10pt}{12pt}\selectfont}
  \newcommand{\fnmarkfont}{\LatoMedium\fontsize{10pt}{11pt}\selectfont}
  \newcommand{\fnmarktextfont}{\LatoMedium\fontsize{8pt}{10pt}\selectfont}
  \newcommand{\fnfont}{\LatoLight\fontsize{8pt}{10pt}\selectfont}
\fi
  \let\bibfont\fnfont  
  \RequirePackage[math-style=ISO, bold-style=ISO]{unicode-math}
  \setmathfont{Garamond-Math.otf}[StylisticSet={7,9}]
%    \end{macrocode}
%
% Font sizes.  We redefine |amsart| ones
%    \begin{macrocode}
 \def\@typesizes{%
    \or{5}{6}\or{5}{6}\or{6}{7}\or{7}{8}\or{8}{10}%
    \or{9.25}{11}% normalsize
    \or{10}{12}\or{\@xipt}{13}\or{\@xiipt}{14}%
    \or{\@xivpt}{17}\or{\@xviipt}{20}}%
  \normalsize \linespacing=\baselineskip
%    \end{macrocode}
%
%
%\section{Page dimensions and paragraphing}
%\label{dimensions}
%
%    \begin{macrocode}
\usepackage[paperwidth=155mm, paperheight=220mm, includeheadfoot,
inner=20mm, outer=17mm, top=20mm, bottom=21mm]{geometry}
\setlength\parskip{0pt}
\setlength\parindent{5mm}
\renewcommand{\arraystretch}{0.9}
\renewenvironment{quotation}{\list{}{%
    \leftmargin5mm \listparindent\normalparindent
    \itemindent\z@
    \rightmargin\leftmargin \parsep\z@ \@plus\p@}%
  \item[]%
}{%
  \vspace\baselineskip
  \endlist
}
\let\endquotation=\endlist % for efficiency
\renewenvironment{quote}{%
  \list{}{\rightmargin5mm\leftmargin\rightmargin}\item[]%
}{%
  \vspace\baselineskip
  \endlist
}
%    \end{macrocode}
%
%\section{Colors}
%\label{sec:colors}
%
% We use DeutscheMuseumRed color
%    \begin{macrocode}
\RequirePackage{xcolor}
\definecolor{DeutschesMuseumRed}{cmyk}{0.36,0.87,0.61,0.33}
%    \end{macrocode}
%
%\section{Front matter}
%\label{sec:front_matter}
%
% \begin{macro}{\subtitle}
% The subtitle
%    \begin{macrocode}
\newcommand\subtitle[1]{\def\@subtitle{#1}\def\shorttitle{#1}}
\subtitle{}
%    \end{macrocode}
% \end{macro}
%
% \begin{macro}{\maketitle}
% And |maketitle| command
%    \begin{macrocode}
\renewcommand\maketitle{%
  \@afterindentfalse
  \ifx\@empty\shortauthors \let\shortauthors\shorttitle
  \else \andify\shortauthors
  \fi
  \par
  \@topnum\z@ % this prevents figures from falling at the top of page
  \thispagestyle{firstpage}%
  \ifx\@title\@empty
  \else{\noindent\color{DeutschesMuseumRed}\hIfont\@title\par}\fi
  \ifx\@subtitle\@empty\else
     {\vspace\baselineskip\noindent\hIIfont\@subtitle\par}%
  \fi
  {\noindent\author@andify\authors\normalsize\normalfont\itshape\authors\vspace\baselineskip\par}%
  \@afterheading
}
%    \end{macrocode}
% \end{macro}
%
% \begin{macro}{\abstract}
% \changes{1.1}{2025/09/04}{Added abstract handling}
% Abstract is an unnumbered section divided from the body
%    \begin{macrocode}
\renewenvironment{abstract}{\section*{\abstractname}}{\medskip}
%    \end{macrocode}
% \end{macro}
%
%\subsection{Sectioning}
%\label{sec:sectioning}
%
% \begin{macro}{\section}
% The standard sectioning
%    \begin{macrocode}
\renewcommand\section{%
  \@startsection{section}{1}{\z@}{-\baselineskip}{1sp}{\hIIIfont}}
%    \end{macrocode}
% \end{macro}
% \begin{macro}{\subsection}
% The standard subsectioning
%    \begin{macrocode}
\renewcommand\subsection{%
  \@startsection{subsection}{2}{\z@}{-\baselineskip}{1sp}{\hIVfont}}
%    \end{macrocode}
% \end{macro}
%
%
%\section{Footnotes}
%\label{sec:footnotes}
%
% \begin{macro}{\footnoterule}
% We do not wnat rules
%    \begin{macrocode}
\def\footnoterule{}
%    \end{macrocode}
% \end{macro}
%
% \begin{macro}{\@makefnmark}
% The footnote marker in the text
%    \begin{macrocode}
\def\@makefnmark{\textsuperscript{{\color{DeutschesMuseumRed}\fnmarktextfont\@thefnmark}}}
%    \end{macrocode}
% \end{macro}
%
% \begin{macro}{\@makefntext}
% The text of a footnote
%    \begin{macrocode}
\def\@makefntext{{\color{DeutschesMuseumRed}\fnmarktextfont\@thefnmark}\quad
  \fnfont}
%    \end{macrocode}
% \end{macro}
%
%
%\section{Page styles}
%\label{sec:page_styles}
%
%    \begin{macrocode}
\RequirePackage{fancyhdr}
%    \end{macrocode}
%
% Standard headers
%    \begin{macrocode}
\renewcommand{\headrulewidth}{0pt}
\renewcommand{\footrulewidth}{0pt}
%    \end{macrocode}
%
% First page style
%    \begin{macrocode}
\fancypagestyle{firstpage}{\fancyhf{}}
%    \end{macrocode}
%
% And the standard page style
%    \begin{macrocode}
\fancypagestyle{standard}{%
  \fancyhf{}%
  \fancyhead[RO]{\foliofont\shorttitle\quad\thepage}%
  \fancyhead[LE]{\foliofont\thepage\quad\shortauthors}%
}
%    \end{macrocode}
%
%
%\section{Bibliography}
%\label{sec:biblio}
%
% We use natbib
%    \begin{macrocode}
\RequirePackage{natbib}
%    \end{macrocode}
%
% \begin{macro}{\cite}
% Our citations are footnotes
%    \begin{macrocode}
\RenewDocumentCommand{\cite}{O{}O{}m}{\footnote{\citealt[#1][#2]{#3}}}
%    \end{macrocode}
% \end{macro}
%
% And some redefinitions
%    \begin{macrocode}
\renewcommand\bibsection{\section*{\refname}}
\renewcommand\refname{Bibliography}
\def\bibpreamble{\setlength\multicolsep{\z@}\begin{multicols}{2}}
\def\bibcleanup{\vskip-\lastskip\end{multicols}}%  
%    \end{macrocode}
%
%\subsection{End of Class}
%\label{end}
%
%
%    \begin{macrocode}
\pagestyle{standard}
\normalsize
\normalfont
%</class>
%    \end{macrocode}
%
%
%\Finale
%\clearpage
%
%\PrintChanges
%\clearpage
%\PrintIndex
%
\endinput
